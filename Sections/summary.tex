\textsc{We are here to "spread"} the news about dandelions and their effects on our environment. Problem A consisted of two sub-tasks: (1) modeling the spread of a single dandelion across a one-hectare plot of land over the course of one year and (2) developing a method to measure the impact of a foreign species in a new environment. 
  
To model the spread of dandelions over a plot of land, we developed a \textit{Seed Agent Model}, which follows a dandelion's life cycle from seed to plant to puffball, including germination, growth, and death. We decided to use the Brownian Motion Process to model random dandelion movement and treat seed growth and death as a stochastic agent-based problem, meaning each seed was treated independently. To make our model adaptable to different regions and climates, we introduced several \textit{hyperparameters} of temperature, light, wind, consumers, and soil nutrients, which are calibrated by location.

After we developed our model, we collected easily accessible data for our hyperparameters from three different climate regions: Clay, NY (temperate); Phoenix, AZ (arid); and Florida Keys, FL (tropical). Our results show that dandelion population growth tends to follow a similar cyclical trend year-over-year, where plant populations spike in their seed dispersal and blooming seasons and remain dormant otherwise, regardless of the region's climate; however, different regional climates resulted in different blooming seasons. Across all three regions, our model yielded a final population of around \(500\) to \(900\) dandelion plants after one year. Lastly, to test the validity of our model, we employed a Monte Carlo sampling simulation (\(\pm 10\%\)) and a sensitivity analysis of all hyperparameters, which revealed that light levels and seed lifespans significantly affect dandelion growth, whereas nutrients found in the soil were not as impactful.

For part two, we developed a "trade-off" model that considers both the benefits and disadvantages of introducing a foreign species. This was done by splitting the perceived impact into two components: harmful ecological and beneficial economic impact. Concerning the foreign species' impact on their new environment, we assessed their effect on native species and the speed at which they spread. To do this, we used a Lotka-Volterra system of differential equations and a logistic regression of population data, respectively. As for the economic potential of the foreign species, we conveyed the societal impact as a standardized monetary value. To do so, we used the bioeconomic Gordon-Schaefer model to represent the profit made from the plant and Sen's Welfare functions to calculate the plant's social benefits and human health impact. 

From there, we applied the second model to three foreign species — dandelions, garlic mustard, and English ivy — and compared each plant's results. If the economic benefit was significantly higher than the ecological impact, we denoted the species as "non-invasive;" otherwise, it was labeled "invasive." By comparing each plant's results, we decided dandelions were invasive species due to their high ecological impact and below-par economic benefit. Similarly, garlic mustard plants were also considered as invasive, whereas English ivy was considered non-invasive. A benefit of this model is its generalizability to almost any type of plant due to our proposed \textit{Data-Score-Divide} (DSD) framework. Using this framework, species can be automatically classified as invasive or non-invasive by a user-determined threshold to represent the balance between ecological and economic impact scores. 

Overall, our agent-based and stochastic models thoroughly represent dandelions in the environment, how they spread, and their role as an invasive species. Easily adaptable, our models can be used on virtually any plot of land and for any type of invasive species. Although they seem irrelevant growing in your yard, the effects of dandelions on the environment can "blow" you away. \#dandelionPRIDE
